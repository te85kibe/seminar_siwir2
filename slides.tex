\documentclass[handout]{beamer}
% \usetheme{Goettingen}

\usepackage[ngerman]{babel}
\usepackage[utf8]{inputenc}
\usepackage[T1]{fontenc}

\usepackage{tikz}
\usepackage{pgfplots}


\usepackage{setspace}

\usepackage{color}
\usepackage{transparent}

\usepackage{alltt}

\usepackage{amsmath, calc}
\usepackage{amsfonts}
\usepackage{amssymb}

\usepackage{bbm}

% frame number
% \setbeamertemplate{footline}[frame number]
% transparent cover
\setbeamercovered{invisible} % or transparent

% define \norm{}
\newcommand{\norm}[1]{\left\lVert#1\right\rVert}

\setbeamertemplate{footline}[text line]{%
  \parbox{\linewidth}{\vspace*{-8pt}Multigrid Seminar\hfill\insertshortauthor\hfill}}
\setbeamertemplate{navigation symbols}{}

\title{Multigrid Seminar}
\subtitle{Simulation and Scientific Computing 2}
\author{Thomas Köster, Nils Kohl}
\date{6. Februar 2014}
\begin{document}
	% \frame{\titlepage}
    
    \begin{frame}
        \frametitle{Outline}
		Optimizations:
		\begin{itemize}
			\item W-Cycles and Successive Overrelaxation (SOR) to speed up convergence
			\pause
			\item Red-Black grid ordering to simplify the parallelization process
			\pause
			\item OpenMP with static scheduling to parallelize smoothing and restriction/interpolation
		\end{itemize}
    \end{frame}    

    \section{Überblick}    	
    \begin{frame}
        \frametitle{Time measurements}
		Measurements with the smallest possible number of cycles to fit the error constraints:
		\center
		\begin{tikzpicture}
			\begin{axis}[
				xlabel=Threads,
				ylabel=Execution Time (in seconds)]

			\addplot[color=red,mark=x] coordinates {
				(1, 5.42)
				(2, 2.87)
				(4, 1.6)
				(8, 1.6)
			};
			\addlegendentry{\tiny{W-Cycle<2,1>, 4 iterations}}
			
			\addplot[color=blue,mark=*] coordinates {
				(1, 6.38)
				(2, 3.32)
				(4, 1.81)
				(8, 1.77)
			};

			\addlegendentry{\tiny{V-Cycle<2,1>, 7 iterations}}

			\addplot[color=green,mark=o] coordinates {
				(1, 7.54)
				(2, 4.03)
				(4, 3.95)
				(8, 2.29)
			};

			\addlegendentry{\tiny{V-Cycle<1,0>, 9 iterations}}

			\end{axis}
		\end{tikzpicture}
    \end{frame}
\end{document}
